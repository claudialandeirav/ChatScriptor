\capitulo{2}{Objetivos del proyecto}

A continuación, se detallan los objetivos que han motivado la realización del proyecto:

\section{Objetivos generales}\label{objetivos-generales}
    \begin{itemize}
		\tightlist		
		\item
		      Desarrollar una aplicación web que permita realizar el mantenimiento y la gestión de chatbots Dialogflow.
		\item
			Mejorar la interfaz gráfica desarrollada por Google para su aplicación web Dialogflow.
		\item
			Aportar nuevas funcionalidades a la aplicación desarrollada.			
	\end{itemize}

\section{Objetivos técnicos}\label{objetivos-tecnicos}
        \begin{itemize}
		\tightlist		
		\item
			Desarrollar una aplicación Flask que realice de manera clara las funciones ya implementadas en la versión original y permita realizar otras nuevas.
            \item 
                Emplear un \textit{framework} para diseño de aplicaciones web como Bootstrap.
            \item 
                Aplicar una estructura software en 3 capas como la arquitectura MVP (\textit{Model-View-Presenter}) para el desarrollo de la aplicación.
		\item
			Utilizar Git como software de control de versiones junto con el uso de la herramienta GitHub.
            \item 
                Aplicar metodologías ágiles como la metodología Scrum en el desarrollo del proyecto.
	\end{itemize}

\section{Objetivos personales}\label{objetivos-personales}
        \begin{itemize}
            \item 
                Poner en práctica la mayor cantidad de conocimientos adquiridos durante el periodo académico desarrollado en la Universidad de Burgos.
            \item 
                Conocer el procedimiento del desarrollo completo de un proyecto software.
            \item 
                Adquirir conocimientos sobre la aplicación de la inteligencia artificial sobre campos como el comercio, la industria o la educación.
            \item 
                Profundizar en el desarrollo software y diseño de interfaces.
        \end{itemize}