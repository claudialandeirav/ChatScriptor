\capitulo{6}{Trabajos relacionados}
En este apartado, se espera encontrar proyectos y trabajos semejantes al realizado. Aunque tras llevar a cabo una exhaustiva búsqueda, no se ha encontrado ningún trabajo similar. 

Por lo que el propósito de las siguientes divisiones es el de mostrar trabajos sobre los que influye el desarrollo de esta mejora de interfaz y de funcionalidades con respecto a Dialogflow.

Se han encontrado diferentes trabajos que utilizan la API Dialogflow, junto con otras funcionalidades, con el fin de crear agentes con funcionamientos concretos. Algunos de los cuales se describen a continuación.

El primero de ellos, es ``\textit{Desarrollo de un Chatbot con Dialogflow en el Marco de las Ciudades Inteligentes}'' realizado por Sergio Francisco Iáñez Gonzálex \cite{ianez2018desarrollo}.

Este escrito, muestra la aplicación de un agente conversacional creado con Dialogflow en el ámbito de las ciudades inteligentes, así como sus funcionalidades dentro de la sociedad actual y su interacción con la misma sin necesidad de pasar por las grandes curvas de aprendizaje que, en algunas situaciones, representa el uso de este tipo de herramientas.


El siguiente trabajo encontrado fue el titulado ``\textit{Interfaz conversacional en Dialogflow para recomendación de películas}'' realizado por Carlos Magán López \cite{magan2019interfaz}.

En este caso, se trata de la elaboración de un asistente capaz de realizar recomendaciones de películas basándose en el uso de algoritmos de recomendación, como filtros colaborativos, utilizando APIs y otras técnicas de \textit{web scraping}\footnote{El \textit{web scraping} \cite{¿QuéEsel14:online} es una forma de extracción de información de páginas web mediante software.} que, unidos a la interfaz conversacional, permiten dar al usuario una serie de elementos cinematográficos adaptados a sus gustos.

Otra publicación encontrada fue ``\textit{Diseño y desarrollo de un asistente interactivo transaccional para la reserva de hoteles}'' realizado por Beatriz Soro Vegas \cite{soro2020diseno}.

Esta memoria, tal y como viene representado en el título, tiene como objetivo desarrollar un agente que permita realizar reservas de alojamientos a los usuarios, añadiendo las funcionalidades de diferentes procesamientos de pagos, haciendo uso de diversas APIs con esas aplicaciones.

La cuestión final es saber en qué se relacionan estos trabajos con este proyecto. La realidad es que al ser agentes creados desde la plataforma de Dialogflow, existe la posibilidad de exportarlos e importarlos en nuestra aplicación, permitiendo utilizar y configurar estos asistentes virtuales, traducirlos o disponer de una mayor claridad a la hora de buscar información que se quiera editar.

Tal y como se explicó en el previo análisis de la aplicación web de Dialogflow, dentro del apartado de \textit{Conceptos teóricos}, concretamente en el punto \textit{3.2 Dialogflow}, sus buscadores son muy poco prácticos y, para este caso de desarrollos, que poseen una gran cantidad de entidades e \textit{intents}, ir comprobando en cuál de ellos se encuentra el término que quiero modificar, es algo increíblemente negativo. 

Además de que la adición de lenguajes en el chatbot puede llevar a tener traducciones semánticas erróneas ya que se queda un poco justa, mientras que si traducimos directamente el chatbot completo con ChatScriptor, debido al uso de tecnologías como el \textit{Natural Language Processing} (NLP), las frases son traducidas siguiendo la intención de dicha oración, interpretándola y buscando la mejor forma de expresarla.